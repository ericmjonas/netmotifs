\documentclass{article}
\usepackage[english,american]{babel}
\usepackage[disable]{todonotes}
\usepackage{subfiles}
\usepackage{grffile}
\usepackage{booktabs}
\usepackage{hyperref}
\usepackage{algorithm}
\usepackage{algorithmic}
\usepackage{amsmath}
\usepackage{subcaption}
\usepackage{multicol}

\title{Neural Circuit Discovery}
\author{Eric Jonas \\ Konrad Kording}

\begin{document}
\maketitle

\listoftodos

\begin{abstract}
  The brain is made out of multiple types of neurons, and a neuron’s
  type affects both its behavior and which other neurons it connects
  to. Emerging techniques (connectomics) allow measuring the
  connectivity matrix between many neurons. Estimating types from
  connectomics is difficult as distance is usually more important than
  type. Here we describe a nonparametric Bayesian technique that
  overcomes this problem and discovers neuron types based on
  connections only. We show that the approach recovers known neuron
  types in the retina, reveals interesting structure in the nervous
  system of c. elegans, and automatically discovers the structure of
  microprocessors. Extracting meaningful structure from connectivity
  data promises to enable new experiments and to constrain theories of
  brain function.

\end{abstract}

\section{Introduction}
``type'' is a recurreing theme in biology
Cell type in bio literature, stem cells, something something

It is only through understanding the types of the system that we can understand
its function. 

\section{Methods}
Stochastic blockmodels long known

We take an infinite extension of the SBM (the IRM ) \todo{cite both}
and modify it to learn the structure of the underlying code. 


Spatial component by performing a GLM-style link function. 

Describe the model

We develop a series of Markov-chain monte carlo scheems to solve for
the posterior, presenting the MAP estimate. To avoid local minima we
slowly anneal, over 1000 iterations \todo{Fix numbers}.

Hyperparameter inference. 

We show that we recover ground truth. 


\section{Results}

\subsubsection{Mouse Retina}
The recent release of connectome-scale data for the mouse retina. 

Look we recover known classes! 

\subsection{Drosophila}

\subsection{C. elegans}


\section{Discussion}


\end{document}
