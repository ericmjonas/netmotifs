\documentclass{article}
\usepackage[english,american]{babel}
\usepackage[disable]{todonotes}
\usepackage{subfiles}
\usepackage{grffile}
\usepackage{booktabs}
\usepackage{hyperref}
\usepackage{algorithm}
\usepackage{algorithmic}
\usepackage{amsmath}
\usepackage{subcaption}
\usepackage{multicol}

\title{Neural Circuit Discovery}
\author{Eric Jonas \\ Konrad Kording}

\begin{document}
\maketitle

\listoftodos

\begin{abstract}
  The brain is made out of multiple types of neurons, and a neuron’s
  type affects both its behavior and which other neurons it connects
  to. Emerging techniques (connectomics) allow measuring the
  connectivity matrix between many neurons. Estimating types from
  connectomics is difficult as distance is usually more important than
  type. Here we describe a nonparametric Bayesian technique that
  overcomes this problem and discovers neuron types based on
  connections only. We show that the approach recovers known neuron
  types in the retina, reveals interesting structure in the nervous
  system of c. elegans, and automatically discovers the structure of
  microprocessors. Extracting meaningful structure from connectivity
  data promises to enable new experiments and to constrain theories of
  brain function.

\end{abstract}

\section{Introduction}
Computing systems, biological or human-engineered, contain computing
elements that can be classified into ``type'', which give
insight into function. 

The advent of connectomics data allows discovery of cell types based
on connectivity information. The rise of clustering had a substantial
impoact on molecular biology, and is now the only way that molecular
biologists are able to deal with high-throughput sequencing
technologies.

[bAyesian nonparametric techniques have been proposed to 
discover structure in data, which could be applied to connecomics data]

Here we describe a Bayesian non-parametric model that can discover
both the cell types and their patterns of interconnection automatically
from connectomics data. 

We apply it to three very-different computing systems:
recently-released mouse retina connectome, the c. elegans connectome,
and a ``connectome'' obtained by reverse-engineering a portion of a
classical microprocessor.

In all cases, we recover known types and suggest possible new types. 


\section{Results}
[add paragraph that summarizes methods and the question]

\subsection{Synthetic Data}
We show our model works by generating synthetic data with known
spatial/connectivity patterns and recovering ground truth, even when
the generating process makes assumptions very different from our
model.

\subsubsection{Mouse Retina}
intro paragraph about retina data

When applied to the mouse retina dataset, we recover
spatially-homogeneous patterns of activity by anlayzing the
connectivity data.

This recovers structure similar to that known by neuroanatomists [ground truth]


\subsection{C. elegans}
intro paragraph to c elegans data

When applied to c. elegans, we segregate ``head and sensor'' interneurons from the spatially-distributed neurons along the body axis. 

\subsection{Microprocessor}
intro paragraph to microprocessor data 

We recover the logical structure of the three primary registers of the
MOS6502 integrated circuit.


\section{Discussion}
We present a machine learning technique that allows cell types to be
disocvered using only connectivity data. 

[Two parts: First is how you suck -- cover the things that are missing
such as adding additional features ]

[second is : how this changes things]

In the future, we plan to extend our model with the ability to handle
other classes of information, including genetic and histochemical
metadata.



\section{Methods}
Stochastic block models assume a hidden or ``latent'' type is associated
wtih each node that ultimately influences its connectivity. 

We extend this class of model to incorporate notions of spatial locality, 
as the connectivity of many neural systems is substantially constrained. 

We perform posterior inference in this model using a series of Markov-chain
Monte Carlo techniques. 

\end{document}
